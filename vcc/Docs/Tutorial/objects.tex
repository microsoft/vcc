\section{Object invariants}
\label{sect:invariants}

Pre- and postconditions allow specification that a function requires
or ensures that data is in a consistent state on entry to or exit from
the function. However, it is usually better practice to associate such
consistency with the data itself.  This is particularly important for
data accessed concurrently by multiple threads (where data must be
kept in a consistent state at all times), but even for
sequential programs enforcing consistency conditions on data reduces
annotation clutter and allows for introduction of abstraction
boundaries.

In VCC, you can associate \Def{object invariants} with compound C types
(\vcc{struct}s and \vcc{union}s).  The invariants of a type describe
how ``good'' objects of that type behave.  In this section and the next,
we consider only the static aspects of this behavior, namely
what the ``good'' states of an object are.  Invariants can also constrain
how how ``good'' objects can change, are covered in
\secref{concurrency}.  


As a first example, consider the following type definition of
\vcc{'\0'}-terminated safe strings implemented with statically
allocated arrays (we'll see dynamic allocation later).

\vccInput[linerange={obj-init}]{c/6.1.safestring.c}

\noindent
The invariant of \vcc{SafeString} states that a good 
\vcc{SafeString} has length not more than \vcc{SSTR_MAXLEN} and is
\vcc{'\0'}-terminated.  Within a type invariant, \vcc{\this} is a pointer
to the current instance of the type (as in the first
invariant), but fields can also be referred to directly (as in the
second invariant). 

Because memory in C is allocated without initialization, no nontrivial
object invariant is guaranteed to hold on allocation. An object that
is known to be in a good state (which implies its invariants hold) is
said to be \Def{closed}, while one that is now known to be in a good
state is said to be \Def{open}. A mutable object is just an open
object owned by the current thread; a \Def{wrapped} object is a closed
object owned by the current thread. A function serving as a
constructor for a type will normally establish invariant and wrap the object:

\vccInput[linerange={init-append}]{c/6.1.safestring.c}

\noindent
For a pointer \vcc{p} of structured type, \vcc{\span(p)} returns the
set of pointers to members of \vcc{p}. Arrays of base types produce
one pointer for each base type component, so in this example,
\vcc{\span(s)} abbreviates the set
\begin{VCC}
  { s, &s->len, &s->content[0], &s->content[1], ..., &s->content[SSTR_MAXLEN] }
\end{VCC}

Thus, the writes clause says that the function can write the fields
of \vcc{s}.  The postcondition says that the function returns with
\vcc{s} wrapped, which implies also that the invariant of \vcc{s}
holds; this invariant is checked when the object is wrapped. (You can
see this check fail by commenting out any of the assignment statements.)

A function that modifies a wrapped object must first unwrap it, make
the necessary updates, and wrap the object again (which causes another
check of the object invariant); this is because VCC does not allow
(nonvolatile) fields of an object while it is closed. Unwrapping an
object adds all of its 
members to the writes set of a function, so such a function has to
report that it writes the object, but does not have to report writing
the fields of the object:

\vccInput[linerange={append-index}]{c/6.1.safestring.c}

\noindent
Finally, a function that only reads an object need not unwrap, and so
need not list it in its writes clause. For example:

\vccInput[linerange={index-out}]{c/6.1.safestring.c}

VCC keeps track of whether an object is closed with the \vcc{\bool}
field \vcc{\closed} (which is a field of every object).  It keeps
track of the owner of an object with the field \vcc{\owner}.
This field is a pointer to an object, which might be a thread.

\begin{VCC}
_(def \bool \wrapped(\object o) {
   return \non_primitive_ptr(o) && o->\closed && o->\owner == \me;
})
_(def \bool \mutable(\object o) {
   if (!\non_primitive_ptr(o)) return \mutable(\embedding(o));
   return  !o->\closed && o->\owner == \me;
}
\end{VCC}
These definitions use several new features:
\begin{itemize} 
\item
When verifying a body of a function VCC assumes that it is being
executed by some particular thread.  The \vcc{\thread} object
representing it is referred to as \vcc{\me}. 
\item
The type \vcc{\object} is much like \vcc{void*}, in the sense
that it is a wildcard for any pointer type. However, while casting a
pointer to \vcc{void *} causes information about its type to be lost,
casting to \vcc{\object} does not. Note that (sadly) \vcc{\object}
includes not just pointers to objects, but also pointers to primitive
types like \vcc{int} that can be fields of objects but are not
first-class objects. What we call ``objects'' are \vcc{\object}s for
which the function \vcc{\non_primitive_ptr} returns \vcc{\true}.
\item
When applied to a pointer that is not an object, \vcc{\embedding(o)}
returns the object of which \vcc{o} is a field. Unlike pointers in C,
VCC pointers include information about the object of which they are a
field. 
\end{itemize}

\subsection{Wrap/unwrap protocol}
\label{sect:wrap-unwrap}
We now consider wrapping and unwrapping in more detail. 

If an object is owned by a thread, only that thread can change its
(nonvolatile) fields (and then only if the object is open), wrap or unwrap
it, or change its owner to another object. Objects are guaranteed to
be closed when they are owned by objects other than threads, and only
closed objects can own other objects. Finally, threads are always
closed, and own themselves. 

The call \vcc{_(unwrap o)}
translates essentially to the following:
\begin{enumerate}
\item \vcc{_(assert \wrapped(o))};
\item \vcc{_(assert \writable(o))}, i.e., that \vcc{o} was either
  listed in the writes clause of the function, or became \vcc{\wrapped}
  after the current function activation started;
\item assume the invariant of \vcc{o}
\item set \vcc{o->\closed = \false};
\item add the span of the object (\ie all its fields) to the writes
  set;
\item set \vcc{\me} to be the owner of any objects owned by \vcc{o};
\item assert that the transition did not violate any invariants of
  \vcc{o} of the form \vcc{_(invariant _(on_unwrap(\this,p)))}.
\end{enumerate}
The operation \vcc{_(wrap o)} does the reverse:
\begin{enumerate}
\item
assert that \vcc{o} is mutable;
\item
assert that all objects whose ownership is to be transfered to \vcc{o}
(to be defined later) are \vcc{\wrapped} and \vcc{\writable};
\item set \vcc{o->\closed = \true};
\end{enumerate}
VCC also provides the syntax 
\begin{VCC}
_(unwrapping o) { ... }
\end{VCC}
which is equivalent to:
\begin{VCC}
_(unwrap o) { ... } _(wrap o)
\end{VCC}

The assert\slash assume desugaring of the \vcc{sstr_append_char()}
function is roughly as follows:

\vccInput[linerange={assert-out}]{c/6.2.safestring_assert.c}

\subsection{Ownership trees}
\label{sect:ownership}

Objects often stand for abstractions that are implemented with more
than just one physical object.  As a simple example, consider our
\vcc{SafeString}, changed to have a dynamically allocated buffer.  The
logical string object consists of the control object holding the
length and the array of bytes holding the content:

\vccInput[linerange={obj-append}]{c/6.3.safestring_dynamic.c}

\noindent
In C the type \vcc{char[10]} denotes an array with exactly 10
elements.  VCC extends that location to allow the type
\vcc{char[capacity]} denoting an array with \vcc{capacity} elements
(where \vcc{capacity} is an expression).  Such types can be only used
in casts (in annotations). For example, \vcc{(char[capacity])content}
means to take the pointer \vcc{content} and interpret it as an \Def{array
object} consisting of \vcc{capacity} elements of type \vcc{char}. 

The invariant of \vcc{SafeString} specifies that it \Def{owns} the
array object. (The use of an array object is necessary; \vcc{char}s
are not first class objects (they can only be fields of other objects)
and so cannot have owners.)
The syntax \vcc{\mine(o1, ..., oN)} is roughly equivalent to
\begin{VCC}
o1->\owner == \this && ... && oN->\owner == \this
\end{VCC}
Conceptually there isn't much difference between \vcc{content} being 
an actual field of the \vcc{SafeString} (as it was in the previous
definition) and it being an array object owned by the \vcc{SafeString}.
In particular, in neither case does a function operating on a \vcc{\wrapped} 
\vcc{SafeString} \vcc{s} have to list \vcc{s->content} (or any other
objects owned by \vcc{s}) in their writes clauses. This is because
modifying \vcc{s->content} requires first unwrapping \vcc{s}, and
doing so adds \vcc{s->content} to the writes set. For example:

\vccInput[linerange={append-alloc}]{c/6.3.safestring_dynamic.c}

\noindent
Let \vcc{cont = (char[s->capacity]) s->content}. 
At the beginning of the function, \vcc{s} is owned by the
current thread (\vcc{\me}) and closed (\ie \vcc{\wrapped}), whereas
(by the string invariant) \vcc{cont} is owned by \vcc{s} (and
therefore closed).  Unwrapping \vcc{s} transfers ownership of 
\vcc{cont} to \vcc{\me}, but \vcc{cont} remains closed.
Thus, unwrapping \vcc{s} makes the string mutable, and \vcc{cont}
wrapped.  Then we unwrap \vcc{cont} (which doesn't own anything, so
the thread gets no new wrapped objects), perform the changes, and wrap
\vcc{cont}.  Finally, we wrap \vcc{s}.  This transfers ownership
of \vcc{cont} from the current thread to \vcc{s}, so \vcc{cont}
is no longer wrapped (but still closed).  Here is
the assert\slash assume translation:

\vccInput[linerange={append-out}]{c/6.4.safestring_dynamic_assert.c}

\noindent
Here, \vcc{\inv(p)} means the (user-defined) invariant of object \vcc{p}.
There are two ownership transfers
of \vcc{cont} to and from \vcc{\me} because \vcc{s} owns \vcc{cont} beforehand,
as specified in its invariant.
However, suppose we had an invariant like the following:
\begin{VCC}
struct S {
  struct T *a, *b;
  _(invariant \mine(a) || \mine(b))
};
\end{VCC}
When wrapping an instance of \vcc{struct S}, VCC wouldn't know which object
to transfer ownership of to the instance. Therefore, 
VCC rejects such invariants, and only allow \vcc{\mine(...)}
as a top-level conjunct in an invariant, unless further annotation is given;
see \secref{dynamic-ownership}.

\subsection{Dynamic ownership}
\label{sect:dynamic-ownership}

When a struct is annotated with \vcc{_(dynamic_owns)} the ownership transfers
during wrapping need to performed explicitly, but \vcc{\mine(...)} can
be freely used in its invariant, including using it under a universal
quantifier.

\vccInput[linerange={obj-set}]{c/6.5.table.c}

\noindent
The invariant of \vcc{struct SafeContainer} states that it owns its underlying array,
as well as all elements pointed to from it.
It also says that there are no duplicates in that array.
Suppose \vcc{o} is a \vcc{SafeContainer} and 
we want to change \vcc{o->strings[idx]},
from \vcc{x} to \vcc{y}.
After such an operation, the container should own whatever it used
to own, minus \vcc{x}, plus \vcc{y}.
To facilitate this, VCC provides a field \vcc{\owns} for each object
with the implicit invariant
\begin{VCC}
_(invariant \this->\closed ==> \forall \object q;
  (q \in \this->\owns <==> q->\owner == \this)
\end{VCC}

That is, for closed \vcc{p}, the set \vcc{p->\owns} contains exactly
the objects that have \vcc{p} as their owner.
The opertions \vcc{_(wrap p)} and \vcc{_(unwrap p)} do not
change \vcc{p->\owns},
so \vcc{_(wrap p)} attempts to transfer to \vcc{p} ownership of all 
object in \vcc{p->\owns}, which (as described previously) causes an
assertion that all of these objects are \vcc{\wrapped} and \vcc{\writable}.

Thus, the usual pattern is to unwrap \vcc{o}, potentially modify
fields of \vcc{o} and \vcc{o->\owns}, and wrap \vcc{o}.
Note that when no ownership transfers are needed, one can just unwrap
and wrap \vcc{O}, without worrying about what \vcc{o} owns.
Here is an example with an ownership transfer:

\vccInput[linerange={set-use}]{c/6.5.table.c}

\noindent
The \vcc{sc_set()} function transfers ownership of \vcc{s} to \vcc{c},
and additionally leaves object initially pointed to by \vcc{s->strings[idx]}
wrapped. Moreover, it promises that this object is \Def{fresh}, \ie
the thread did not own it directly before.  This can be used at a call
site:

\vccInput[linerange={use-out}]{c/6.5.table.c}

\noindent
In the contract of \vcc{sc_add} the string \vcc{s} is mentioned
in the writes clause, but in the postcondition we do not say it's wrapped.
Thus, asserting \vcc{\wrapped(s)} after the call fails.
On the other hand, asserting \vcc{\wrapped(o)} fails before the call,
but succeeds afterwards.
Additionally, \vcc{\wrapped(c)} holds before and after as expected.

\subsection{Ownership domains}

The \Def{sequential ownership domain} of an object \vcc{o} (written
\vcc{\domain(o)}) consists of \vcc{o} along with\footnote{ The domains
  of the objects owned by \vcc{o} are included only if \vcc{o} is not
  declared as \vcc{_(volatile_owns)}; see \secref{concurrency}.} the
union of the ownership domains of all objects owned by \vcc{o}.  In
other words, it's the set of objects that are transitively owned by
\vcc{o}. For most purposes, it is useful to think of \vcc{o} as
``containing'' all of \vcc{\domain(o)}; indeed, if \vcc{o1 != o2} and
neither \vcc{o1} nor \vcc{o2} are in the other's \vcc{\domain}, their
\vcc{\domain}s are necessarily disjoint. In particular, if \vcc{o1}
and \vcc{o2} are owned by threads then (because threads own
themselves) \vcc{o1} and \vcc{o2} are necessarily disjoint.

Writability of \vcc{o} gives a thread potential access to all of
\vcc{\domain(o)}: writability allows the thread to unwrap \vcc{o},
which makes writable both the fields of \vcc{o} and any objects that were
owned by \vcc{o}. Conversely, a call to a function that does not list
a wrapped object \vcc{o} in its writes clause is guaranteed to leave 
all of \vcc{\domain(o)} unchanged\footnote{
  This applies to nonvolatile fields of objects in
  the domain; volatile fields might change silently (see section \secref{concurrency}).
}. However, VCC will only reason about the unchangedness of
\vcc{\domain(o)} if it is explicitly brought to its attention, as in
the following example:

\begin{VCC}
void f(T *p) 
  _(writes p) { ... }
...
T *p, *q, *r;
_(assert \wrapped(q) && q != p)
_(assert q \in \domain(q))
_(assert r \in \domain(q))
f(p);
\end{VCC}
\noindent
The second and third assertions bring to VCC's attention that as long
as \vcc{q} is not explicitly unwrapped or included in the writes
clause of a function call, \vcc{r} and its fields will not change.

\subsection{Simple sequential admissibility}
\label{sect:admissibility0}

Until now we've ignored the issue of constraints on the form of object
invariants. When VCC checks an atomic update to the state, it checks
only the invariants of those objects that are actually
updated. Constraints on the invariants of other objects are needed to
guarantee that invariants of closed unupdated objects are also preserved.

Fortunately, the most common case is trivial: if an invariant of
\vcc{o} mentions only (nonvolatile) fields of objects in
\vcc{\domain(o)}, the invariant is necessarily admissible.
More sophisticated kinds of invariants are discussed in \secref{inv2}.

\subsection{Type safety}
\label{sect:type-safety}

In modern languages like Java, C\#, and ML, where memory consists of a
collection of typed objects. Programs in these languages allocate
objects (as opposed to memory), and the type of an object remains
fixed until the object is destroyed. Moreover, a non-null reference to
an object is guaranteed to point to a ``valid'' object. But in C, a
type simply provides a way to interpret a sequence of bytes; nothing
prevents a program from having multiple pointers of different types
pointing into the same physical memory, or having a non-null
pointer point into an invalid region of memory.

That said, most C programs really do access memory using a strict type
discipline and tacitly assume that their callers do
so also. For example, if the parameters of a function are a pointer to
an \vcc{int} and a pointer to a \vcc{char}, we shouldn't have to worry
about crazy possibilities like the \vcc{char} aliasing with the second
half of the \vcc{int}. (Without such assumptions, we would have to
provide explicit preconditions to this effect.)  On the other hand, if
the second parameter is a pointer to an \vcc{int}, we do consider the
possibility of aliasing (as we would in a strongly typed language).
Moreover, since in C objects of structured types literally contain
fields of other types, if the second argument were a struct that had
a member of type \vcc{int}, we would have to consider the possibility
of the first parameter aliasing that member. 

To support this kind of antialiasing by default, VCC essentially
maintains a typed view of memory, as follows. Each object has a
\vcc{\bool} field \vcc{\valid}. Each object requires a (possibly
empty) set of real memory addresses for its representation; this set
of addresses constitutes the ``footprint'' of an objecct. VCC
maintains the invariant that in any state, the footprints of distinct
valid objects do not overlap. It maintains this in a very simple way:
it only makes an object valid by simultaneously making invalid a set
of objects, the union of whose footprints include the footprint of the
newly valid object. This invariant allows us to reason completely in
terms of objects, without every having to worry about aliasing between
one object and another\footnote{
This does not completely remove the need to occasionally reason about
address arithmetic, since VCC sometimes has to reason about the
relationships between the address of, say, and object \vcc{o} and an
object \vcc{o-f} nested within \vcc{o}.
}. Only valid fields (and purely local variables) can be
accessed by a program.

There are rare situations where a program needs to change the type of
memory, i.e., make one object invalid while making valid an object
that aliases with it. The most common example is in the memory
allocator, which needs to create and destroy objects of arbitrary
types from arrays of bytes in its memory pool. Therefore, VCC includes
annotations (explained in \secref{reint}) that explicitly change
object validity (and are in fact the only means to do so).  Thus,
while your program can access memory using pretty much arbitrary types
and typecasting, doing so is likely to require additional
annotations. But for most programs, checking type safety is completely
transparent, so you don't have to worry about it.

